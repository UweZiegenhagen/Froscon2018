\documentclass[ngerman]{scrreprt}

\usepackage[utf8]{inputenc}
\usepackage[T1]{fontenc}
\usepackage{babel}
\usepackage{microtype}

\usepackage[a4paper,landscape]{geometry}

\usepackage{blindtext}
\usepackage{booktabs}
\usepackage{graphicx} % JPG, PNG und PDF

\newcommand{\uzi}{Uwe Andreas Ziegenhagen}

\newcommand{\fk}[1]{\textbf{\textit{#1}}}


\title{Hallo Froscon!}
\author{Uwe Ziegenhagen}

\usepackage{hyperref}
\hypersetup{
    bookmarks=true,                     % show bookmarks bar
    unicode=false,                      % non - Latin characters in Acrobat’s bookmarks
    pdftoolbar=true,                        % show Acrobat’s toolbar
    pdfmenubar=true,                        % show Acrobat’s menu
    pdffitwindow=false,                 % window fit to page when opened
    pdfstartview={FitH},                    % fits the width of the page to the window
    pdftitle={My title},                        % title
    pdfauthor={Author},                 % author
    pdfsubject={Subject},                   % subject of the document
    pdfcreator={Creator},                   % creator of the document
    pdfproducer={Producer},             % producer of the document
    pdfkeywords={keyword1, key2, key3},   % list of keywords
    pdfnewwindow=true,                  % links in new window
    colorlinks=true,                        % false: boxed links; true: colored links
    linkcolor=blue,                          % color of internal links
    filecolor=blue,                     % color of file links
    citecolor=blue,                     % color of file links
    urlcolor=blue                        % color of external links
}



\begin{document}

\listoftables

\clearpage

\begin{tabular}{|l|r|c|p{5cm}|} \hline
Hallo & Welt & Ich bin & eine kleine Tabelle in meinem Dokument \\ \hline
Hallo Bonn 2018 & Welt Froscon & Ich bin & eine kleine Tabelle in meinem Dokument \\ \hline
Hallo & Welt & Ich bin & eine kleine dreizeilige Tabelle in meinem Dokument \\ \hline
\end{tabular}

\vspace*{2cm}

\begin{tabular}{lrcp{5cm}} \toprule[2pt]
\textbf{Spalte 1} & \textbf{Spalte 2} & \textbf{Spalte 3} & \textbf{Spalte 4} \\ \midrule
Hallo & Welt & Ich bin & eine kleine Tabelle in meinem Dokument \\ 
Hallo Bonn 2018 & Welt Froscon & Ich bin & eine kleine Tabelle in meinem Dokument \\ 
Hallo & Welt & Ich bin & eine kleine dreizeilige Tabelle in meinem Dokument \\ \bottomrule[3pt]
\end{tabular}

\begin{table}[htb]
\caption{Hallo, ich bin eine Tabelle}
\begin{tabular}{lrcp{5cm}} \toprule[2pt]
\textbf{Spalte 1} & \textbf{Spalte 2} & \textbf{Spalte 3} & \textbf{Spalte 4} \\ \midrule
Hallo & Welt & Ich bin & eine kleine Tabelle in meinem Dokument \\ 
Hallo Bonn 2018 & Welt Froscon & Ich bin & eine kleine Tabelle in meinem Dokument \\ 
Hallo & Welt & Ich bin & eine kleine dreizeilige Tabelle in meinem Dokument \\ \bottomrule[3pt]
\end{tabular}
\end{table}


\captionof{figure}{Hallo, ich bin eine Tafdsfsbelle}
\begin{tabular}{lrcp{5cm}} \toprule[2pt]
\textbf{Spalte 1} & \textbf{Spalte 2} & \textbf{Spalte 3} & \textbf{Spalte 4} \\ \midrule
Hallo & Welt & Ich bin & eine kleine Tabelle in meinem Dokument \\ 
Hallo Bonn 2018 & Welt Froscon & Ich bin & eine kleine Tabelle in meinem Dokument \\ 
Hallo & Welt & Ich bin & eine kleine dreizeilige Tabelle in meinem Dokument \\ \bottomrule[3pt]
\end{tabular}



\end{document}

