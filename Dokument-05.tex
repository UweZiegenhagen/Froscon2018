\documentclass[ngerman]{scrreprt}

\usepackage[utf8]{inputenc}
\usepackage[T1]{fontenc}
\usepackage{babel}
\usepackage{microtype}

\usepackage{blindtext}

\newcommand{\uzi}{Uwe Andreas Ziegenhagen}

\newcommand{\fk}[1]{\textbf{\textit{#1}}}


\title{Hallo Froscon!}
\author{Uwe Ziegenhagen}

\begin{document}

\maketitle

\tableofcontents

\chapter{Froscon-Einführung}

\section{Einführung}

\subsection{Sankt Augustin}

\subsubsection{Nabel der Welt}


\blindtext[1]

\begin{itemize}
	\item Hallo
	\item Froscon
	\item Ich 
	\item bin 
	\item eine 
	\item Aufzählung
\end{itemize}

\begin{enumerate}
	\item Hallo
	\item Froscon
	
\begin{enumerate}
	\item Hallo
	\item Froscon
	\item Ich 
	\item bin 
	\item eine 
	\item Aufzählung
\end{enumerate}	
	
	\item Ich 
	\item bin 
	\item eine 
	\item Aufzählung
\end{enumerate}


\begin{description}
\item[Rot] \blindtext
\item[Grün] ist eine Farbe
\item[Blau] ist eine Farbe
\end{description}

\uzi

Hallo, dies ist ein \textbf{fettgedrucktes} Wort.

Hallo, dies ist ein \textit{kursives} Wort.

Hallo, dies ist ein \textbf{\textit{kursives}} Wort.

Hallo, ich bin ein \fk{Wort}

\end{document}